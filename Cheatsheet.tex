\documentclass[a4paper]{article}

% Packages for layout and math
\usepackage[margin=0.2cm]{geometry} % Narrow margins to fit more content
\usepackage{multicol} % For the 2-column layout
\usepackage{amsmath,amssymb}
\usepackage{enumitem}
\usepackage{titlesec}

% Global font and spacing tweaks (smaller than plain \tiny)
\linespread{0.8} % tighten line spacing a bit
\tiny % start from tiny
\makeatletter
\renewcommand\normalsize{%
  \@setfontsize\normalsize{6pt}{7pt}% smaller base size than default tiny
}
\makeatother
\normalsize

\setlength{\parskip}{0pt}
\setlength{\parindent}{0pt}
\setlist[itemize]{leftmargin=*,nosep}

% Simple block macro (no framed environment → avoids blank pages)
\newcommand{\Block}[1]{%
  \par\noindent
  \fbox{%
    \parbox{0.98\linewidth}{#1}% slightly narrower box to help LaTeX break lines
  }%
  \par\vspace{0.15ex}%
}

\begin{document}

\begin{multicols}{2}

\Block{%
\textbf{Cross-Validation (Cross-Val)}\\
Objective: partition data into folds to estimate model generalization.\\
Cross-val tests model on new data. Split data into folds, train on most, test on one, repeat. Avg performance is reliable.\\
\textbf{Pros:} Better perf estimate than single split; Detects overfitting.\\
\textbf{Cons:} More time/compute for many folds/large data; Tricky for time-series.
}

\Block{%
\textbf{Convex Optimization (Convex-Opt)}\\
Objective: minimize convex $f(x)$ s.t. convex $g_i(x)\le0$, affine $h_j(x)=0$.\\
Convex opt finds global best in bowl-shaped space. Used in SVMs, regression.\\
\textbf{Pros:} Guarantees global min; Efficient solvers.\\
\textbf{Cons:} Not all probs convex, need approx; Heavy for large probs.
}

\Block{%
\textbf{Solving LP/QP (Quick Guide)}\\
Form: LP $\min_x c^\top x$ s.t.\ $Ax\le b$, $A_{\rm eq}x=b_{\rm eq}$; QP $\min_x \tfrac12 x^\top Qx+c^\top x$ s.t.\ linear constraints, $Q\succeq0$.\\
Steps: (1) collect vars into $x$, write all linear constraints; (2) simplify equalities by eliminating fixed vars; (3) small/medium: use interior-point LP/QP solver; large/sparse: use (projected) gradient / coordinate descent. Example QP: L2-regularized least squares $\min_w \tfrac12\|Xw-y\|_2^2+\tfrac\lambda2\|w\|_2^2$ is a convex QP; closed form $(X^\top X+\lambda I)w=X^\top y$ or solve with CG. KKT: primal feas., dual feas., stationarity, complementary slackness certify optimality.
}

\Block{%
\textbf{Gradient Descent (GD)}\\
Update: $\theta^{t+1}=\theta^t-\eta\nabla f(\theta^t)$.\\
GD updates params opposite grad of loss to min errors.\\
\textbf{Pros:} Simple to implement; Good for convex.\\
\textbf{Cons:} Slow on large data (full set/step); Stuck in local min for non-convex.
}

\Block{%
\textbf{Stochastic Gradient Descent (SGD)}\\
Update (single sample $i$): $\theta^{t+1}=\theta^t-\eta\nabla\ell_i(\theta^t)$.\\
SGD like GD but updates w/one random point, faster/noisier.\\
\textbf{Pros:} Faster on large data; Escapes local min via noise.\\
\textbf{Cons:} Noisy, erratic; Needs LR scheduling.
}

\Block{%
\textbf{Mini-batch Gradient Descent}\\
Update (batch $B_t$): $\theta^{t+1}=\theta^t-\eta\frac1{|B_t|}\sum_{i\in B_t}\nabla\ell_i(\theta^t)$.\\
Mini-batch GD updates w/small batches, balances GD/SGD.\\
\textbf{Pros:} Faster than GD, less noisy than SGD; GPU-efficient.\\
\textbf{Cons:} Batch size tuning needed; Can stuck in local min.
}

\Block{%
\textbf{Data Augmentation}\\
Objective: minimize loss on augmented dataset $\min_\theta\sum_{(x,y)\in\mathcal D_{\text{aug}}}\ell(y,f_\theta(x))$.\\
Data aug mods existing ex (rotate, noise) for robust models.\\
\textbf{Pros:} More data w/o collect; Better gen, esp images.\\
\textbf{Cons:} May add unreal data; Compute-heavy in train.
}

\Block{%
\textbf{Lagrangian}\\
Formulation: $L(w,\lambda,\nu)=f(w)+\sum_i\lambda_i g_i(w)+\sum_j\nu_j h_j(w)$, $\lambda_i\ge0$.\\
Lagrangian combines obj func w/constraints via mults for opt pts.\\
\textbf{Pros:} Solves eq/ineq constraints; Base for SVMs.\\
\textbf{Cons:} Complex math; Needs KKT checks.
}

\Block{%
\textbf{Dual Lagrangian}\\
Dual problem: $\max_{\lambda\ge0,\nu} g(\lambda,\nu)$ where $g(\lambda,\nu)=\inf_w L(w,\lambda,\nu)$.\\
Dual reformulates primal, often easier, esp kernels.\\
\textbf{Pros:} Simplifies computation in many cases; Enables kernel trick for non-linear problems.\\
\textbf{Cons:} May increase complexity for some formulations; Requires careful handling of dual variables.
}

\Block{%
\textbf{K-Nearest Neighbors (KNN)}\\
Objective: no parametric minimization; prediction $\hat y(x)=\operatorname{mode}\{y_i:x_i\in\mathcal N_k(x)\}$.\\
KNN classifies a new data point based on the majority label of its 'k' closest neighbors in the training data, using distance metrics like Euclidean.\\
\textbf{Pros:} Simple and intuitive, no training phase needed; Works well for non-linear data.\\
\textbf{Cons:} Slow for large datasets (computes distances at prediction time); Sensitive to irrelevant features and noise.
}

\Block{%
\textbf{Naive Bayes}\\
Objective: $\min_\theta -\sum_i\log p_\theta(y_i)p_\theta(x_i|y_i)$ with $p(x|y)=\prod_j p(x_j|y)$.\\
Naive Bayes is a probabilistic classifier that applies Bayes' theorem, assuming features are independent, to predict class probabilities.\\
\textbf{Pros:} Fast and efficient, especially for high-dimensional data like text; Performs well even with the 'naive' independence assumption.\\
\textbf{Cons:} Assumption of feature independence often unrealistic; Struggles with zero-probability issues (use smoothing).
}

\Block{%
\textbf{Linear Discriminant Analysis (LDA)}\\
Objective: $\min_{\mu_k,\Sigma,\pi_k} -\sum_i\log(\pi_{y_i}\mathcal N(x_i|\mu_{y_i},\Sigma))$.\\
LDA projects data onto a lower-dimensional space to maximize class separability, assuming Gaussian distributions and equal covariances.\\
\textbf{Pros:} Good for dimensionality reduction while preserving class info; Computationally efficient.\\
\textbf{Cons:} Assumes normality and equal covariances, which may not hold; Linear boundaries only.
}

\Block{%
\textbf{Logistic Regression}\\
Objective: $\min_w \sum_i \log(1+\exp(-y_i w^\top x_i))+\lambda\|w\|_2^2$.\\
Logistic Regression models the probability of binary outcomes using a sigmoid function on a linear combination of features.\\
\textbf{Pros:} Interpretable coefficients show feature importance; Handles binary and multi-class (via one-vs-rest).\\
\textbf{Cons:} Assumes linear decision boundaries; Sensitive to multicollinearity.
}

\Block{%
\textbf{Support Vector Machines (SVM)}\\
Soft-margin primal: $\min_{w,b,\xi}\tfrac12\|w\|^2 + C\sum_i\xi_i$ s.t.\ $y_i(w^\top x_i+b)\ge 1-\xi_i$, $\xi_i\ge0$.\\
SVM finds the hyperplane that best separates classes with the maximum margin, using support vectors.\\
\textbf{Pros:} Effective in high-dimensional spaces; Robust to overfitting with proper regularization.\\
\textbf{Cons:} Computationally intensive for large datasets; Sensitive to choice of kernel and parameters.
}

\Block{%
\textbf{Kernel SVM}\\
Dual: $\max_\alpha \sum_i\alpha_i-\tfrac12\sum_{ij}\alpha_i\alpha_j y_i y_j K(x_i,x_j)$ s.t.\ $\sum_i\alpha_i y_i=0$, $0\le\alpha_i\le C$.\\
Kernel SVM extends SVM to non-linear data by mapping to higher dimensions via kernels (e.g., RBF) without explicit transformation.\\
\textbf{Pros:} Handles complex, non-linear boundaries; Versatile with different kernels.\\
\textbf{Cons:} More computationally expensive; Risk of overfitting if kernel not chosen well.
}

\Block{%
\textbf{Linear Regression}\\
Objective: $\min_w \sum_i (y_i-w^\top x_i)^2 = \min_w\|Xw-y\|_2^2$.\\
Linear Regression fits a line to data by minimizing squared errors, predicting outputs as a linear combination of inputs.\\
\textbf{Pros:} Simple and interpretable; Fast to train.\\
\textbf{Cons:} Assumes linearity; poor for complex relationships; Sensitive to outliers.
}

\Block{%
\textbf{Ridge Regression}\\
Objective: $\min_w \sum_i (y_i-w^\top x_i)^2 + \lambda\|w\|_2^2$.\\
Ridge Regression adds L2 regularization to linear regression to shrink coefficients and handle multicollinearity.\\
\textbf{Pros:} Reduces overfitting and stabilizes estimates; Good for correlated features.\\
\textbf{Cons:} Includes all features (no selection); Bias introduced by regularization.
}

\Block{%
\textbf{Lasso Regression}\\
Objective: $\min_w \sum_i (y_i-w^\top x_i)^2 + \lambda\|w\|_1$.\\
Lasso Regression uses L1 regularization, which can set some coefficients to zero for feature selection.\\
\textbf{Pros:} Performs automatic feature selection; Handles multicollinearity.\\
\textbf{Cons:} Can be unstable with highly correlated features; Bias like Ridge.
}

\Block{%
\textbf{Kernel Ridge}\\
Objective: $\min_\alpha \|K\alpha-y\|_2^2 + \lambda\alpha^\top K\alpha$.\\
Kernel Ridge combines Ridge regression with kernels for non-linear fitting.\\
\textbf{Pros:} Captures non-linear patterns; Regularization prevents overfitting.\\
\textbf{Cons:} Computationally heavy for large data; Kernel tuning required.
}

\Block{%
\textbf{Support Vector Regression (SVR)}\\
Soft-margin primal: $\min_{w,b,\xi,\xi^*}\tfrac12\|w\|^2 + C\sum_i(\xi_i+\xi_i^*)$ s.t.\ $|y_i-w^\top x_i-b|\le\epsilon+\xi_i$, $\xi_i,\xi_i^*\ge0$.\\
SVR adapts SVM for regression, finding a function that deviates from actual values by at most epsilon.\\
\textbf{Pros:} Robust to outliers; Effective in high dimensions.\\
\textbf{Cons:} Sensitive to parameter choice (C, epsilon); Slow for large datasets.
}

\Block{%
\textbf{Kernel SVR}\\
Objective: same SVR primal in feature space; dual uses kernel $K(x_i,x_j)$.\\
Kernel SVR uses kernels for non-linear regression in SVR.\\
\textbf{Pros:} Handles complex non-linear data; Flexible with kernels.\\
\textbf{Cons:} Increased complexity and compute; Overfitting risk.
}

\Block{%
\textbf{Polynomial Regression}\\
Objective: $\min_w \sum_i (y_i-w^\top\phi(x_i))^2$ with polynomial features $\phi(x)$ (e.g.\ degree $d$).\\
Polynomial Regression fits higher-degree polynomials to capture non-linear trends.\\
\textbf{Pros:} Simple extension of linear regression; Good for curved relationships.\\
\textbf{Cons:} Prone to overfitting with high degrees; Extrapolation can be poor.
}

\Block{%
\textbf{K-Means}\\
Objective: $\min_{\{c_k\},\{z_i\}}\sum_i\|x_i-c_{z_i}\|^2$ with $z_i\in\{1,\dots,k\}$.\\
K-Means partitions data into k clusters by minimizing within-cluster variance, assigning points to nearest centroids.\\
\textbf{Pros:} Simple and scalable; Fast convergence.\\
\textbf{Cons:} Needs k specified; sensitive to initialization; Assumes spherical clusters.
}

\Block{%
\textbf{Gaussian Mixture Model (GMM)}\\
Objective: $\min_{\pi_k,\mu_k,\Sigma_k} -\sum_i\log\sum_k\pi_k\mathcal N(x_i|\mu_k,\Sigma_k)$ s.t.\ $\pi_k\ge0$, $\sum_k\pi_k=1$.\\
GMM models data as a mixture of Gaussian distributions, using EM to estimate parameters.\\
\textbf{Pros:} Handles elliptical clusters and soft assignments; Probabilistic outputs.\\
\textbf{Cons:} Slower than K-Means; sensitive to init; Assumes Gaussian components.
}

\Block{%
\textbf{Perceptron}\\
Objective (implicit): minimize misclassification by updates $w\leftarrow w+y_i x_i$ on errors.\\
Perceptron is a single-layer neural network for linear classification, updating weights on errors.\\
\textbf{Pros:} Basic building block of NNs; Converges for linearly separable data.\\
\textbf{Cons:} Only linear; no hidden layers; Doesn't handle XOR-like problems.
}

\Block{%
\textbf{Multi-Layer Perceptron (MLP)}\\
Objective: $\min_\theta \sum_i \ell(y_i,f_\theta(x_i))+\lambda\|\theta\|^2$ (cross-entropy/MSE).\\
MLP adds hidden layers to Perceptron for non-linear learning via backpropagation.\\
\textbf{Pros:} Universal approximator for functions; Handles complex data.\\
\textbf{Cons:} Prone to overfitting; needs regularization; Black-box; hard to interpret.
}

\Block{%
\textbf{Convolutional Neural Networks (CNN)}\\
Objective: same as MLP, $\min_\theta \sum_i \ell(y_i,f_\theta(x_i))+\\Omega(\theta)$, with conv/pooling layers.\\
CNN uses convolutional layers for feature extraction, ideal for grid data like images.\\
\textbf{Pros:} Excellent for spatial hierarchies (e.g., images); Parameter sharing reduces compute.\\
\textbf{Cons:} Requires large data and GPU; Overfits without augmentation.
}

\Block{%
\textbf{Sequential Minimal Optimization (SMO)}\\
Objective: solve SVM dual $\max_\alpha \sum_i\alpha_i-\tfrac12\sum_{ij}\alpha_i\alpha_j y_i y_j K_{ij}$ s.t.\ $0\le\alpha_i\le C$, $\sum_i\alpha_i y_i=0$ by updating two $\alpha$ at a time.\\
SMO solves the SVM dual by updating two Lagrange multipliers at a time while keeping constraints satisfied.\\
\textbf{Pros:} Efficient for large SVM problems; Avoids large QP solvers; Works well with kernels.\\
\textbf{Cons:} More complex to implement than simple GD; Speed depends on heuristics for picking pairs.
}

\Block{%
\textbf{RANSAC}\\
Objective: find parameters $\theta$ maximizing inliers while being robust to outliers.\\
RANSAC repeatedly samples minimal subsets, fits a model, and counts inliers to find a robust fit under many outliers.\\
\textbf{Pros:} Very robust to outliers; Simple concept; Works well for geometric vision tasks.\\
\textbf{Cons:} Needs many iterations if inlier ratio low; Requires thresholds and max-iter tuning.
}

\Block{%
\textbf{Expectation Maximization (EM)}\\
Objective: maximize $\ell(\theta)=\sum_i\log\sum_Z p(x_i,Z;\theta)$. E-step: compute $Q(\theta|\theta^{old})=\mathbb E_{Z|X,\theta^{old}}[\log p(X,Z;\theta)]$. M-step: $\theta^{new}=\arg\max_\theta Q(\theta|\theta^{old})$.\\
EM maximizes a latent-variable likelihood by alternating: E-step (compute posteriors/expectations) and M-step (maximize expected complete log-likelihood).\\
\textbf{Pros:} Handles missing/latent variables naturally; Closed-form updates for models like GMM.\\
\textbf{Cons:} Converges only to local maxima; Can be slow; Sensitive to initialization.
}

\Block{%
\textbf{Dimensionality Reduction}\\
Objective: find mapping $z=f(x)$ preserving variance, distances, or class info.\\
Dimensionality reduction maps high-dim data to lower-dim space while preserving structure (variance, distances, or class info).\\
\textbf{Pros:} Reduces storage and computation; Helps visualization and denoising.\\
\textbf{Cons:} May discard useful information; Choice of method and target dim is non-trivial.
}

\Block{%
\textbf{Feature Selection}\\
Objective: $\min_w \sum_i \ell(y_i,f_w(x_i))+\lambda\|w\|_0$ (NP-hard, relaxed to L1 or greedy).\\
Feature selection chooses a subset of input features (filter, wrapper, embedded methods) instead of transforming them.\\
\textbf{Pros:} Improves interpretability; Can reduce overfitting and training time.\\
\textbf{Cons:} Search can be expensive; Risk of discarding informative but weak features.
}

\Block{%
\textbf{Linear Dimensionality Reduction}\\
Objective: $\max_W \operatorname{tr}(W^\top SW)$ s.t.\ $W^\top W=I$ (PCA), or ratio of between/within scatter (LDA).\\
Linear DR finds projections $z = W^\top x$ that keep most variance or class separation (e.g., PCA, LDA).\\
\textbf{Pros:} Simple and fast; Often has eigenvalue/eigenvector closed forms.\\
\textbf{Cons:} Only captures linear structure; Fails on curved manifolds (non-linear relations).
}

\Block{%
\textbf{Singular Value Decomposition (SVD)}\\
Objective: $X=U\Sigma V^\top$, best rank-$k$ approx: $\min_{\operatorname{rank}(X_k)\le k}\|X-X_k\|_F^2$ solved by $X_k=U_k\Sigma_k V_k^\top$.\\
SVD: $X = U\Sigma V^\top$, with orthogonal $U,V$ and singular values in $\Sigma$.\\
\textbf{Pros:} Basis of PCA and low-rank approximations; Optimal rank-$k$ approximation in Frobenius norm.\\
\textbf{Cons:} Expensive on very large matrices; Often needs truncated or randomized SVD.
}

\Block{%
\textbf{Principal Component Analysis (PCA)}\\
Objective: $\max_W \sum_i\|W^\top x_i\|^2$ s.t.\ $W^\top W=I$ (find top eigenvectors of covariance).\\
PCA finds directions of maximum variance (eigenvectors of covariance, or top right-singular vectors of $X$).\\
\textbf{Pros:} Unsupervised linear DR; Decorrelates features; Often improves downstream methods.\\
\textbf{Cons:} Components are linear and not label-aware; Sensitive to scaling and outliers.
}

\Block{%
\textbf{Kernel PCA}\\
Objective: same as PCA in feature space; eigendecomposition of centered kernel matrix $K$.\\
Kernel PCA applies PCA in an implicit feature space using a kernel matrix instead of the covariance of raw features.\\
\textbf{Pros:} Captures non-linear structure; Works with same kernels as Kernel SVM.\\
\textbf{Cons:} Needs storing and eigendecomposing $N\times N$ kernel matrix; Less interpretable than standard PCA.
}

\Block{%
\textbf{Whitening}\\
Objective: enforce $\text{Cov}(z)=I$ via linear transform $z=\Sigma^{-1/2}U^\top x$ after PCA.\\
Whitening transforms data so that it has zero mean and identity covariance (decorrelated, unit variance). Often done after PCA.\\
\textbf{Pros:} Removes linear correlations; Useful preprocessing for some models and ICA.\\
\textbf{Cons:} Can amplify noise in low-variance directions; Requires good covariance estimate.
}

\Block{%
\textbf{Looking at Learning Curves}\\
Objective: plot train and test error vs. training set size or training iterations to diagnose bias/variance.\\
Learning curves plot train and validation error vs. training set size or epochs.\\
\textbf{Pros:} Helps diagnose high-bias vs. high-variance; Guides whether to get more data or change model complexity.\\
\textbf{Cons:} Requires repeated training; Interpretation can be ambiguous with noisy curves.
}

\Block{%
\textbf{PCA, Kernel PCA, Whitening}\\
Objectives: PCA max variance in input; Kernel PCA max variance in feature space; Whitening enforce $\text{Cov}(z)=I$.\\
\textbf{Similarities:} All linear transforms in some space; Used for preprocessing and dimensionality reduction.\\
\textbf{Differences:} PCA linear in input space; Kernel PCA non-linear via kernels; Whitening rescales to identity covariance (often after PCA) instead of just keeping top-variance directions.
}

\end{multicols}

\end{document}
